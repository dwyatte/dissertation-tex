\documentclass[dwyatte_dissertation.tex]{subfiles} 
\begin{document}

\sloppy

\chapter{General Discussion}
\section{Summary of principal results}
The work described throughout this thesis has centered around how prediction is used in sensory processes such as object recognition and prolonged learning. The work was heavily motivated by the LeabraTI (TI: Temporal Integration) framework (Chapter \ref{chap:leabrati}) which leverages the laminocolumnar structure of the neocortex \cite{Mountcastle97,BuxhoevedenCasanova02,HortonAdams05} to learn to predict temporally structured sensory inputs. Predictive learning in the LeabraTI framework is made possible by temporally interleaving predictions and sensory processing across the same populations of neurons so that powerful error-driven learning mechanisms \cite{OReillyMunakata00,OReillyMunakataFrankEtAl12} can be used to compute a prediction error that can be learned against to minimize the difference between predictions and sensory events over time.

LeabraTI relies on a 10 Hz prediction-sensation period as its core ``clock cycle'', suggested to correspond to the widely studied alpha rhythm observable across posterior cortex using scalp EEG \cite{PalvaPalva07,HanslmayrGrossKlimeschEtAl11,VanRullenBuschDrewesEtAl11}. Chapter \ref{chap:pleast} investigated the role of the alpha rhythm in prediction by using an entrainment paradigm \cite{SchroederLakatosKajikawaEtAl08,CalderoneLakatosButlerEtAlInPress} in which stimuli were presented rhythmically at 10 Hz so that predictions and sensory information could be interleaved regularly at the optimal rate proposed by LeabraTI. The experiment made use of three-dimensional objects that required integration over multiple sequential views to extract their three-dimensional structure. Thus, relatively rapid predictive learning mechanisms that operate over subsequent 100 ms periods could be leveraged to optimally encode the the objects. The spatial coherence between views and temporal onset of each view were independently manipulated to determine their effect on stimulus encoding quality and the putative role of the alpha rhythm in predictive processing.

The results of the Chapter \ref{chap:pleast} experiment indicated that spatial coherence and predictable temporal onset of each stimulus in an entraining sequence enhanced discriminability of a subsequently presented probe stimulus. Oscillatory analyses indicated strong bilateral alpha power and phase coherence modulation as a function of stimulus predictability. Specifically, spatial predictability of entraining stimuli suppressed alpha power with a lower degree of phase misalignment relative to unpredictable stimuli. Temporally predictable entraining stimuli had the opposite effect, with increased alpha power and phase alignment, indicating successful entrainment. Importantly, phase alignment due to temporal predictability remained elevated compared to temporally unpredictable stimuli during a 200 ms blank period between the entraining sequence and probe, indicating that the effects of temporal predictability could persist without exogenous entrainment. In addition to these bilateral main effects, right hemisphere sites exhibited synergistic effects of combined spatial and temporal probe predictability on EEG amplitude and 10 Hz phase coherence approximately 200 ms after probe onset. 

Overall, the results of the Chapter \ref{chap:pleast} experiment support the basic claims put forth by the LeabraTI framework. The predictable 10 Hz presentation rate of the entraining sequence improved encoding of the target object, enhancing discriminability for the subsequent probe stimulus. This finding was accompanied by increased alpha phase alignment that remained elevated until the onset of the probe, which is necessary for ensuring that the probe event is processed when the brain is expecting sensory information and not when it is generating a prediction. %However, LeabraTI predicts that spatially predictable sequences presented at a regular temporal interval should elicit a synergistic effect on behavioral measures due to the multiple prediction-sensation cycles that successfully integrate visuospatial information at optimal temporal intervals \cite[see also]{DohertyRaoMesulamEtAl05,RohenkohlGouldPessoaEtAl14}. A synergistic effect of combined spatial and temporal predictability was demonstrated for EEG results, but not for behavioral measures.

The Chapter \ref{chap:bpleast} experiment...

\section{Outstanding issues}
% TI error term?

% things that could be brought up
% predictions for bpleast if it was an EEG expt
% reiterate general improvements of experiments
%		and modeling

% are people actually predicting? don't need to, LeabraTI assumes prediction is implicit, when alpha is entrained -- if not, prediction is not reliable
% Stronger 10 Hz effects without attention

%%%
% save for general discussion
%%%
% * LeabraTI: why do low frequency oscillations convey information about spatial predictability
%	  % Low frequency oscillations seem to capture spatial stuff too, not just temporal stuff although it's hard to know for sure without a better expt
%
%	  % 5 Hz better for temporal prediction time window than 10 Hz? Again, at least given current data, but hard to know
% * 5 Hz vs 10 Hz
% *		Is 5 Hz functionally different? Or just a subharmonic that is better suited for significance given the experimental parameters (200 ms blank)
% * relation btwn osc and behave effects
% * 	5 Hz Super bonus related to IE since it is the only cross-over interaction
% * 	5 Hz pow related to accuracy? seems plausible, follows same order
%
% * what about prediction for other modalities -- 5 Hz highlighted for speech, fundamentally different? slower predictions needed? vanrullen notes no 10 Hz temporal aliasing

% TODO: talk about randy's bottom-up alpha suppression 

\section{Conclusions}

\bibliographystyle{apa}
\bibliography{ccnlab}

\end{document}