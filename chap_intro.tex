\documentclass[dwyatte_dissertation.tex]{subfiles} 
\begin{document}

\chapter{Introduction}
\label{chap:intro}

\section{Sensory predictions across space and time}
The brain is often framed as a general purpose ``prediction machine'' \cite{HawkinsBlakeslee04,Clark13}. In this framework, the sole evolved function of the neocortex is to minimize error in its representation of predictions about the physical world. This distillation of function is central to a number of models of neocortical function \cite[e.g.,]{DayanHintonNealEtAl95,RaoBallard99,LeeMumford03,Friston05,GeorgeHawkins09}, but is surprisingly often overlooked in psychology and neuroscience investigations of sensory processing. For example, most experiments are designed to measure evoked responses to a randomly chosen, isolated stimulus under the tacit assumption that response variability is irrelevant noise that should be averaged out across many presentations. Computational models of perceptual processing often operate under similar assumptions in which stimuli are presented as random ``snapshots'' from which some common set of features should be learned to minimize representational variability across presentations (e.g., {\nopcite{RiesenhuberPoggio99,SerreOlivaPoggio07,OReillyWyatteHerdEtAl13}; although see \nopcite{Foldiak91}, for a notable exception). These experimental and modeling assumptions stand in contrast to the event structure of the physical world, which is highly structured from one moment to the next. It could be the case that response variability is not simply due to noise, but is related to meaningful predictive processing that captures this spatiotemporal structure \cite{ArieliSterkinGrinvaldEtAl96,WilderJonesAhmedEtAl13,FischerWhitney14}.

There are a number of important questions that need answered to fully characterize prediction and its role in sensory processing. What are the mechanisms responsible for making predictions? Computationally, there is a fundamental tradeoff in making decisions about and generating actions from the constant stream sensory information to versus actively generating predictions about what will happen next. Do standard mechanisms balance these tradeoffs or is there special purpose, dissociable machinery specifically for predictive processing. Another line of questioning is concerned with how the brain knows \textit{when} to make predictions. Prediction requires integrating information over some time frame and using the result to drive the actual prediction, but when should integration start? And for how long? 

The goal of this thesis is to develop a line of research designed to provide answers to some of these questions and raise others. The work is largely predicated on a modeling framework referred to as \textit{LeabraTI} (Temporal Integration), an extension of standard Leabra cortical learning algorithms \cite{OReillyMunakata00,OReillyMunakataFrankEtAl12} that describes how prediction is accomplished in biological neural circuits. The framework brings together a large number of independent findings from the systems neuroscience literature to describe exactly how multiple interacting mechanisms trade off prediction with sensory processing and learn associations across temporally extended sequences of input. 

The fundamental proposal of LeabraTI is that time in the brain is discretized into roughly 100 ms intervals that allow interleaving predictions and sensory processing over time. These intervals correspond to individual cycles of the widely observed alpha rhythm over posterior cortical areas (\nopcite{KlimeschSausengHanslmayr07}; \abbrevnopcite{MathewsonLlerasBeckEtAl11}). The temporal interleaving of prediction and sensory processing allows standard error-driven learning to operate 

\textbf{TODO}

\section{Brief review of spatial- and temporal-related processing}
% probably best just to frame it as ``there isn't a lot of work on prediction'' so we turn to literature on temporal attention and spatial sequences for review

% space
% what about section on spatial information in sequences?
% previous research on spatial ordering during learning
% * ThorntonKourtzi02 -- faces -- dynamic primes needed to properly activate learned memory trace -- single image is not enough
% * * See also BalasSinha09,VuongTarr,ChuangVuongBulthoff12
% * * GavornikBear14 -- in mice, learned ABCD sequence invokes larger V1 response than DCBA but not in control groups where random sequences were learned
% * Surprise 10 Hz presentation rate finding?

% time
% nobre, etc.

\section{Thesis organization} 
% visual domain, but general across modalities -- a proxy



% TODO



\bibliographystyle{apa}
\bibliography{ccnlab}
\end{document}