\documentclass[dwyatte_dissertation.tex]{subfiles} 
\begin{document}

\chapter{Neural model of spatiotemporal prediction for object recognition}

\section{Introduction}
TODO

\section{Methods}

\subsection{Model architecture}

The model consisted of four layers (Figure \ref{fig:v1_v2}) whose parameters are described in detail in the following paragraphs. Two of the layers contained columnar substructure which was necessary for implementing learning using the LeabraTI framework (Chapter \ref{chap:leabrati}). To simplify the overall LeabraTI computation, only superficial (Layers 2 and 3) and deep (Layer 6) neuron subtypes are explicitly modeled. Projections between these neuron populations correspond to the descending Layer 5 $\rightarrow$ Layer 6 synapses in the brain, which is assumed to be plastic, and the ascending Layer 6 $\rightarrow$ Layer 4 transthalamic synapses which are assumed to be relatively stable and nonplastic.

% task fig
\begin{figure}[h!]
\begin{center}
\end{center}
\caption{}{}
\label{fig:v1_v2}
\end{figure}

\textbf{Retina and V1 preprocessing:} Input was provided to the model via a 24x24 topographic filter bank that preprocessed images offline from the model proper. This preprocessing step is consistent with a large class of biological models describing object recognition in cortex \cite[e.g.,]{RiesenhuberPoggio99,SerreOlivaPoggio07,OReillyWyatteHerdEtAl13} and in the case of the present model, represents visual processing from the level of the retina through V1 simple cells \cite{HubelWiesel62}. Grayscale bitmap images were scaled to 24x24 pixels and convolved with Gabor filters at four orientations (0$^\circ$, 45$^\circ$, 90$^\circ$, and 135$^\circ$) and two polarities (off-on and on-off) producing a 24x24x4x2 set of inputs. Each Gabor filter was implemented as 6x6 pixel kernel, with a wavelength $\lambda$ = 6 and Gaussian width terms of $\sigma_x$ = 1.8 and $\sigma_y$ = 1.2. A static nonlinearity was applied to the output of the filtering step in the form of a modified \textit{k}-Winners-Take-All (\textit{k}WTA) inhibitory competition function that reduced activation across the 4x2 filter bank to the equivalent of \textit{k} = 1 fully active units \cite[see][Supporting Information]{OReillyWyatteHerdEtAl13}.

\textbf{Primary visual layers:} 24x24 topographic layer arranged into groups of 4x2 units (4608 total units), decomposed into superficial and deep neuron subtypes. Each superficial unit received the output of the retina/V1 preprocessing step. \textit{k}WTA inhibition for superficial units was set to 60\% of the average of the top \textit{k} active units compared to the average of all other superficial units with each 4x2 unit group using a value of \textit{k} = 2. Deep units received from 5x5 columns of superficial units (200 inputs per deep unit) integrated into a single value that was used as the additional context input channel for each superficial unit.

\textbf{Secondary visual layers:} 6x6 topographic layer arranged into groups of 7x7 units (1764 total units), also decomposed into superficial and deep neuron subtypes. Each superficial unit received from 8x8 topographical neighborhoods of early visual columns (512 afferents per unit) and sent back reciprocal connections with the same topography. \textit{k}WTA for superficial units was set to 60\% of the average of the top \textit{k} active units compared to the average of all other superficial units with each and 15\% activity within each unit group. Deep units received from 3x3 columns of superficial units (441 inputs per deep unit) integrated into a single value that was used as the additional context input channel for each superficial unit.

\textbf{Output layer:} 10x10 layer (100 total units) without unit group or columnar substructure. Each unit received a full projection from secondary visual columns (1764 afferents per unit) and fully projected back to all columns. A \textit{k}WTA value of \textit{k} = 1 was used to enforce a localist representation. The localist representation is a computational simplification that allowed an identity readout of lower-level features without population decoding similar to that provided by inferior temporal (IT) neurons  \cite{HungKreimanPoggioEtAl05,LiCoxZoccolanEtAl09}. 

\subsection{LeabraTI learning algorithm}

LeabraTI was implemented as an extension of the standard Leabra algorithm which is described in detail in \incite{OReillyMunakata00} and \incite{OReillyMunakataFrankEtAl12}. Standard Leabra learning operates across two phases: a \textit{minus} phase that represents the system's expectation for a given input and a \textit{plus} phase, representing observation of the outcome. The difference between the minus and plus phases, along with additional self-organizing mechanisms, is used in computing the synaptic weight update function at the end of each plus phase. 

LeabraTI extends standard Leabra learning by interleaving its minus and plus phases over temporally contiguous input sequences. In standard Leabra, the minus phase depends on clamped inputs from the sensory periphery to drive the expectation while the plus phase uses clamped outputs from other neural systems to drive the outcome. In LeabraTI, the minus phase expectation is not driven by the sensory periphery, but instead by lagged context represented by deep (Layer 6) neurons. During the plus phase, driving potential shifts back to the sensory periphery. Deep neurons' context is also updated after each plus phase.

LeabraTI was only used to update the synaptic weights between superficial and deep synapses. Inter-areal feedforward and feedback projections bifurcate from the local column, directly synapsing disparate populations of superficial neurons and thus weight updates in these cases were handled by standard Leabra equations. In computing the weight update, standard Leabra uses the difference in firing rate between sending units in the minus phase and  (\textit{x}) and sending and receiving (\textit{y}) units in the plus phase:

\begin{align*}
Leabra:~\Delta w_{ij} = x^-(y^+ - y^-)
\end{align*}

In LeabraTI, deep neurons are considered to be the receiving units but are only active during the minus phase. The most logical delta signal in this case is the plus phase state of the microcolumn, which in the simplified model used here is the firing rate of superficial neurons:

\begin{align*}
LeabraTI:~\Delta w_{ij} = super^+_{t-1}(super^+_{t} - deep^-_{t})
\end{align*}

One consequence of LeabraTI's interleaving of minus and plus phases over time is that the initial minus phase in an input sequence does not have access to the previous moment's (\textit{t} - 1) context. Even if there was lagged context available, it would represent information from a prior, possibly unrelated input sequence. To address this, weight updates are disabled for the first minus-plus phase pair, and enabled thereafter. Presumably this process is facilitated by a neural mechanism that is sensitive to the repetition of inputs over time such as acetylcholine \cite{ThielHensonMorrisEtAl01,ThielHensonDolan02}

% like an SRN but rotated so that the context only needs to be pre-computed

% context filtered through learned synaptic weights
% stable cx-thal wts

%Time in the model is discretized into 100 ms frames coinciding with Layer 5 neurons' alpha spectrum bursting. Each frame contains both Leabra minus and plus phases, with the minus phase driven by deep contextual inputs (no external inputs) and the plus phase driven by the sensory periphery (external inputs), coinciding with thalamic bursting. Full recurrent dynamics are enabled in both phases among superficially connected layers, which are hypothesized to give rise to their differentially high gamma power reported in the literature. Layer 5 context is processed through the Layer 5 $\rightarrow$ Layer 6 synaptic weights at the end of every plus phase in an automatic pacemaker-fashion which is similarly hypothesized to be the mechanism behind deep layers' alpha coherence.

\subsection{Training and testing environment}

\section{Results}

\section{Discussion}

\bibliographystyle{apa}
\bibliography{ccnlab}
\end{document}